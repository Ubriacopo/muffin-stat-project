\newpage
\section{Abstract}
??


\section{Introduction}
There is no doubt that machine learning solutions are taking an important place in everyday applications.
Among those many applications some may seem trivial yet require attention to be solved.

Being able to distinguish Muffins and Chihuahuas, while more on the toy-like nature, is one of such problems.
The task itself is widely solved by many powerful tools nowadays but, as for the goal of this project, we
want to see how close we can get to one of those better trained and engineered models
by simply applying methods and rules of thumb when designing one ourselves.

The problem we want to solve falls under the \textbf{binary classification problems} category,
and it is required to build a solution with Neural Networks.

The results discussed are based on best practices and empirical evidence gathered from the scientific.


\section{Machine and environment information}
# todo: sistema e vedi se effettivamente mettere qualcosa di tanto tecnico
In order to replicate at best the results obtained for the project here are the specs.

\subsection{Machine Information}

All the computations have been run locally.
The system mounts Ubuntu 22.04.4 LTS (Desktop) while the Hardware components are:

\begin{itemize}
    \item GPU: NVIDIA GeForce RTX3070Ti (6GB VRAM)
    \item CPU: AMD® Ryzen 7 5800x 8-core processor × 16
    \item RAM: 32 GB (2x16GB) DDR4
\end{itemize}


\subsection{Development Environment}
To make full use of the GPU power, which is known to be very effective for machine learning given its
capability of working on matrices very efficiently, we had to use the CUDA drivers released by NVIDIA.
For the project we ran on CUDA 11.8.

As it was required to use Keras to develop the project, from the latest release 3.0, we had to choose
the backend that handles the calculations (as Keras is built around those tecnologies).
In the past the only choice was to use \textbf{Tensorflow}, we opted for \textbf{PyTorch} instead as it
is more widespread in the research community.

A list briefly describing the various technologies used can be here^ ## todo reference at end of paper.


\subsection{Brief explanation on the process}
# todo? Fare una speigazione del processo di sviluppo?