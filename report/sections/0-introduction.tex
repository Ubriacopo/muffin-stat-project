\newpage


\section{Introduction}\label{sec:introduction}
Machine learning solutions are taking an important place in everyday applications.
Among the tasks we want to solve some may seem trivial.
Being able to distinguish Muffins and Chihuahuas, which is more of a toy-like problem, is among those.
The task itself is widely solved by many powerful and more general purpose tools nowadays but,
as for the goal for this project, we want to see how close in terms of performance we can get to one of those better engineered solutions.

What we are facing \textbf{binary classification problems} category,
and by the project's requirements it is requested to build a solution using Neural Networks.

The results discussed are based on best practices and empirical evidence gathered from the scientific community.


\section{Machine and environment information}\label{sec:machine-and-environment-information}
In order to replicate at best the results obtained for the project we give a brief on the machine used
and its environments.

\subsection{Machine Information}\label{subsec:machine-information}

All the computations have been run locally.
The system mounts Ubuntu 22.04.4 LTS (Desktop) while the Hardware components are:

\begin{itemize}
    \item GPU: NVIDIA GeForce RTX3070Ti (6GB VRAM)
    \item CPU: AMD Ryzen 7 5800x 8-core processor x16
    \item RAM: 32 GB (2x16GB) DDR4
\end{itemize}

\subsection{Development Environment}\label{subsec:development-environment}
To make full use of the GPU power, which is very effective for machine learning, CUDA drivers were needed. The project ran on CUDA 11.8.

For designing the neural networks \textit{Keras} was used. Since its latest major release (3.0)
the backend, which handles the calculations, is selectable.
\textit{PyTorch} was selected as it is very popular among the research community.

Libraries and other references are listed in the Github repository\cite{Fichera_Muffin_vs_Chihuahua_2024}.